%%%%%%%%%%%%%%%%%%%%%%%%%%%%%%%%%%%%%%%%%%%%%%%%%%%%%%%%%%%%%%%
%% BRIEF VERSION OF OXFORD THESIS TEMPLATE FOR CHAPTER PREVIEWS

%%%%% CHOOSE PAGE LAYOUT
% format for PDF output (ie equal margins, no extra blank pages):
\documentclass[a4paper,nobind]{templates/ociamthesis}

% add hyperref package with links hidden %
\usepackage[colorlinks=false,pdfpagelabels,hidelinks]{hyperref}

% add float package to allow manual control of figure positioning %
\usepackage{float}

%UL set section header spacing
\usepackage{titlesec}
% 
\titlespacing\subsubsection{0pt}{24pt plus 4pt minus 2pt}{0pt plus 2pt minus 2pt}

% UL 30 Nov 2018 pandoc puts lists in 'tightlist' command when no space between bullet points in Rmd file
\providecommand{\tightlist}{%
  \setlength{\itemsep}{0pt}\setlength{\parskip}{0pt}}
 
% UL 1 Dec 2018, fix to include code in shaded environments
\usepackage{color}
\usepackage{fancyvrb}
\newcommand{\VerbBar}{|}
\newcommand{\VERB}{\Verb[commandchars=\\\{\}]}
\DefineVerbatimEnvironment{Highlighting}{Verbatim}{commandchars=\\\{\}}
% Add ',fontsize=\small' for more characters per line
\usepackage{framed}
\definecolor{shadecolor}{RGB}{248,248,248}
\newenvironment{Shaded}{\begin{snugshade}}{\end{snugshade}}
\newcommand{\AlertTok}[1]{\textcolor[rgb]{0.94,0.16,0.16}{#1}}
\newcommand{\AnnotationTok}[1]{\textcolor[rgb]{0.56,0.35,0.01}{\textbf{\textit{#1}}}}
\newcommand{\AttributeTok}[1]{\textcolor[rgb]{0.77,0.63,0.00}{#1}}
\newcommand{\BaseNTok}[1]{\textcolor[rgb]{0.00,0.00,0.81}{#1}}
\newcommand{\BuiltInTok}[1]{#1}
\newcommand{\CharTok}[1]{\textcolor[rgb]{0.31,0.60,0.02}{#1}}
\newcommand{\CommentTok}[1]{\textcolor[rgb]{0.56,0.35,0.01}{\textit{#1}}}
\newcommand{\CommentVarTok}[1]{\textcolor[rgb]{0.56,0.35,0.01}{\textbf{\textit{#1}}}}
\newcommand{\ConstantTok}[1]{\textcolor[rgb]{0.00,0.00,0.00}{#1}}
\newcommand{\ControlFlowTok}[1]{\textcolor[rgb]{0.13,0.29,0.53}{\textbf{#1}}}
\newcommand{\DataTypeTok}[1]{\textcolor[rgb]{0.13,0.29,0.53}{#1}}
\newcommand{\DecValTok}[1]{\textcolor[rgb]{0.00,0.00,0.81}{#1}}
\newcommand{\DocumentationTok}[1]{\textcolor[rgb]{0.56,0.35,0.01}{\textbf{\textit{#1}}}}
\newcommand{\ErrorTok}[1]{\textcolor[rgb]{0.64,0.00,0.00}{\textbf{#1}}}
\newcommand{\ExtensionTok}[1]{#1}
\newcommand{\FloatTok}[1]{\textcolor[rgb]{0.00,0.00,0.81}{#1}}
\newcommand{\FunctionTok}[1]{\textcolor[rgb]{0.00,0.00,0.00}{#1}}
\newcommand{\ImportTok}[1]{#1}
\newcommand{\InformationTok}[1]{\textcolor[rgb]{0.56,0.35,0.01}{\textbf{\textit{#1}}}}
\newcommand{\KeywordTok}[1]{\textcolor[rgb]{0.13,0.29,0.53}{\textbf{#1}}}
\newcommand{\NormalTok}[1]{#1}
\newcommand{\OperatorTok}[1]{\textcolor[rgb]{0.81,0.36,0.00}{\textbf{#1}}}
\newcommand{\OtherTok}[1]{\textcolor[rgb]{0.56,0.35,0.01}{#1}}
\newcommand{\PreprocessorTok}[1]{\textcolor[rgb]{0.56,0.35,0.01}{\textit{#1}}}
\newcommand{\RegionMarkerTok}[1]{#1}
\newcommand{\SpecialCharTok}[1]{\textcolor[rgb]{0.00,0.00,0.00}{#1}}
\newcommand{\SpecialStringTok}[1]{\textcolor[rgb]{0.31,0.60,0.02}{#1}}
\newcommand{\StringTok}[1]{\textcolor[rgb]{0.31,0.60,0.02}{#1}}
\newcommand{\VariableTok}[1]{\textcolor[rgb]{0.00,0.00,0.00}{#1}}
\newcommand{\VerbatimStringTok}[1]{\textcolor[rgb]{0.31,0.60,0.02}{#1}}
\newcommand{\WarningTok}[1]{\textcolor[rgb]{0.56,0.35,0.01}{\textbf{\textit{#1}}}}

%UL 2 Dec 2018 add a bit of white space before and after code blocks
\renewenvironment{Shaded}
{
  \vspace{10pt}%
  \begin{snugshade}%
}{%
  \end{snugshade}%
  \vspace{8pt}%
}
%UL 2 Dec 2018 reduce whitespace around verbatim environments
\usepackage{etoolbox}
\makeatletter
\preto{\@verbatim}{\topsep=0pt \partopsep=0pt }
\makeatother

%UL 28 Mar 2019, enable strikethrough
\usepackage[normalem]{ulem}

%UL use soul package for correction highlighting
\usepackage{soul}
\usepackage{xcolor}
\newcommand{\ctext}[3][RGB]{%
  \begingroup
  \definecolor{hlcolor}{#1}{#2}\sethlcolor{hlcolor}%
  \hl{#3}%
  \endgroup
}
\soulregister\ref7
\soulregister\cite7
\soulregister\autocite7
\soulregister\textcite7
\soulregister\pageref7

% user-included things with header_includes or in_header will appear here
% kableExtra packages will appear here if you use library(kableExtra)

%%%%% SELECT YOUR DRAFT OPTIONS
% Three options going on here; use in any combination.  But remember to turn the first two off before
% generating a PDF to send to the printer!

% This adds a "DRAFT" footer to every normal page.  (The first page of each chapter is not a "normal" page.)

% This highlights (in blue) corrections marked with (for words) \mccorrect{blah} or (for whole
% paragraphs) \begin{mccorrection} . . . \end{mccorrection}.  This can be useful for sending a PDF of
% your corrected thesis to your examiners for review.  Turn it off, and the blue disappears.

%%%%% BIBLIOGRAPHY SETUP
% Note that your bibliography will require some tweaking depending on your department, preferred format, etc.
% The options included below are just very basic "sciencey" and "humanitiesey" options to get started.
% If you've not used LaTeX before, I recommend reading a little about biblatex/biber and getting started with it.
% If you're already a LaTeX pro and are used to natbib or something, modify as necessary.
% Either way, you'll have to choose and configure an appropriate bibliography format...

% The science-type option: numerical in-text citation with references in order of appearance.
% \usepackage[style=numeric-comp, sorting=none, backend=biber, doi=false, isbn=false]{biblatex}
% \newcommand*{\bibtitle}{References}

% The humanities-type option: author-year in-text citation with an alphabetical works cited.
% \usepackage[style=authoryear, sorting=nyt, backend=biber, maxcitenames=2, useprefix, doi=false, isbn=false]{biblatex}
% \newcommand*{\bibtitle}{Works Cited}

%UL 3 Dec 2018: set this from YAML in index.Rmd
\usepackage[style=numeric-comp, sorting=none, backend=biber, doi=false, isbn=false]{biblatex}
\newcommand*{\bibtitle}{References}

% This makes the bibliography left-aligned (not 'justified') and slightly smaller font.
\renewcommand*{\bibfont}{\raggedright\small}

% Change this to the name of your .bib file (usually exported from a citation manager like Zotero or EndNote).

%%%%% YOUR OWN PERSONAL MACROS
% This is a good place to dump your own LaTeX macros as they come up.

% To make text superscripts shortcuts
	\renewcommand{\th}{\textsuperscript{th}} % ex: I won 4\th place
	\newcommand{\nd}{\textsuperscript{nd}}
	\renewcommand{\st}{\textsuperscript{st}}
	\newcommand{\rd}{\textsuperscript{rd}}

%%%%% THE ACTUAL DOCUMENT STARTS HERE
\begin{document}

%%%%% CHOOSE YOUR LINE SPACING HERE
% This is the official option.  Use it for your submission copy and library copy:
\setlength{\textbaselineskip}{22pt plus2pt}
% This is closer spacing (about 1.5-spaced) that you might prefer for your personal copies:
%\setlength{\textbaselineskip}{18pt plus2pt minus1pt}

% UL: You can set the general paragraph spacing here - I've set it to 2pt (was 0) so
% it's less claustrophobic
\setlength{\parskip}{2pt plus 1pt}

% Leave this line alone; it gets things started for the real document.
\setlength{\baselineskip}{\textbaselineskip}

% all your chapters and appendices will appear here
\hypertarget{introduction}{%
\chapter*{Introduction}\label{introduction}}
\addcontentsline{toc}{chapter}{Introduction}

\adjustmtc
\markboth{Introduction}{}

INSERT COMMENT: Per meeting on 3/16, this is the structure that Emily suggested: 1 para = description of big pic prob;
Brief summary (1-3 para) = prior lit most relevant to diss \& gap i'm filling through this research; 1 para on what this dissertation is composed of \& what it seeks to do = in chapter 1 will address X, in chap 2 will address Y -\\
Chapter 1 = other interventions have been done, BUT they haven't done this; Chapter 2 = there are some potential negative effects of these interventions. Transition to chapter 1

\hypertarget{description-of-big-picture-prob}{%
\section{Description of big picture prob}\label{description-of-big-picture-prob}}

Women's labor force participation has increased exponentially over the past several decades \autocite{Goldin2006a,Statistics2020}, rising from 32\% to 57\% between 1960 and 2018 (where women here are defined as 16 years or older) \autocite{Statistics2020,Blau2017,Eagly2019}, while men's participation has decreased over the same period (from 82\% to 69\%). As a result, the gender gap in labor force participation fell to a 12\% difference. Additionally, women have been increasingly entering male-dominated occupations \autocite{Blau2013,Reskin2009,England2010}.

Despite incredible progress towards gender equality (e.g., women's suffrage, a reversal of the gender education gap, women's increased participation in the labor market) \autocite{Goldin2014,Goldin2006a,Goldin2006,Blau2010,Blau2013,Blau2014,Bianchi2012,Sayer2005}, gender gaps in the labor market persist \autocite{Blau2017,Goldin2014,Hegewisch2014,Bertrand2001,Blau2014,Levanon2016,Blau2006b,Blau2006a}. One of the most highly cited and tangible metrics for gender disparities in the labor market is the gender wage gap \autocite{Blau2000,Blau2017,Nyhus2012,McGee2015,Goldin2014,Hegewisch2014,Bertrand2001,Blau2006b}. Recent unadjusted estimates suggest women earn only 79.3\% of what men earn \autocite{Blau2017}.

In the past, classic human capital variables (e.g., gender gaps in education and work experience) explained a large proportion of the gender wage gap (e.g., 27\% in 1980). As women's education and labor force experience has increased over time \autocite{Goldin2006a}, the impact of these variables on the gender wage gap has decreased (e.g., 8\% in 2010) \autocite{Blau2017}. Since women's labor market progress has stalled over the past two decades \autocite{Blau2006b,Goldin2014}, identifying and understanding the factors that perpetuate gender differences in labor market outcomes is crucial for achieving gender equality in the long-term.

\hypertarget{previous-research-and-goals-of-the-current-dissertation}{%
\section{Previous research and goals of the current dissertation}\label{previous-research-and-goals-of-the-current-dissertation}}

\hypertarget{previous-research}{%
\subsection{Previous research}\label{previous-research}}

To that end, one variable that has been explored extensively within the economics literature is competitiveness, both in terms of the choice to compete {[}cites{]} and response to competitions (e.g., performance during competition) {[}cites{]}. These studies typically find that women choose to compete less than men and that they tend to respond less to competition (that is, their performance does not significantly increase to the extent that men's performance does). Since research suggests these gender differences in competitiveness may have implications for real-world economic outcomes {[}cites{]}, researchers began exploring options to increase competitiveness, including enacting gender quotas \autocite{Niederle2013,Sutter2016}, replacing other-competition with self-competition {[}cites{]}, and relaxing pressure during competitions \autocite{Shurchkov2012}, among many others \autocite[see][ for a review]{Niederle2017b}. However, one intervention that has not been explored is offering individuals the opportunity to prepare before entering a competition. We expected that preparation may serve as a viable intervention for increasing women's competitiveness (i.e., choice to compete), since it stands to reason that preparation may increase confidence in one's performance and/or reduce perceptions of risk of competition entry, factors that have been well-established as contributors to the gender gap in competitiveness {[}cites{]}.

\hypertarget{current-dissertation}{%
\subsection{Current dissertation}\label{current-dissertation}}

We address this gap in the literature on gender differences in competitiveness through a series of experiments in Chapter 1 where we offer participants variations of the opportunity to prepare (i.e., knowledge of preparation, limited opportunity to prepare, and unlimited opportunity to prepare) and test whether the gender gap in competitiveness is eliminated. Our research in Chapter 1 also had the explicit goal of identifying whether there are any gender differences in preparation, given the well-established gender differences in risk attitudes and confidence. Given the evidence from Chapter 1 that there are in fact gender differences in preparation, but experimentally tested whether the gender difference in preparation would be exacerbated in competitive settings, such that women would be especially likely to prepare before entering a competitive, relate to non-competitive setting. Chapter 2 also seeks to understand the gender difference in preparation further by exploring whether women are more likely to perceive that they prepare less than others relative to men, especially in competitive settings. Overall, the findings from this set of experiments across both chapters have implications not only for the literature on gender differences in competitiveness, but also contribute more generally to our understanding of how interventions intended to reduce gender differences may have negative downstream consequences (e.g., potential opportunity costs of (over)preparing). In addition, we discover a novel gender difference in preparation, which we encourage future research to explore further.


%%%%% REFERENCES

% JEM: Quote for the top of references (just like a chapter quote if you're using them).  Comment to skip.
% \begin{savequote}[8cm]
% The first kind of intellectual and artistic personality belongs to the hedgehogs, the second to the foxes \dots
%   \qauthor{--- Sir Isaiah Berlin \cite{berlin_hedgehog_2013}}
% \end{savequote}

\setlength{\baselineskip}{0pt} % JEM: Single-space References

{\renewcommand*\MakeUppercase[1]{#1}%
\printbibliography[heading=bibintoc,title={\bibtitle}]}

\end{document}
